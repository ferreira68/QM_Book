% ------------------------------------------------------------------------
% Book Outline
% ------------------------------------------------------------------------
\documentclass[oneside,12pt]{book}
\usepackage[ansinew]{inputenc}
\usepackage{graphicx}
\usepackage{dcolumn}
\usepackage{amssymb}
% ----------------------------------------------------------------------
%  stuff to make margins smaller because book margins put too little
%  on a page
% ----------------------------------------------------------------------
\addtolength{\oddsidemargin}{-.75in}	
\addtolength{\evensidemargin}{-.75in}	
\addtolength{\textwidth}{1.5in}	
\addtolength{\topmargin}{-.875in}	
\addtolength{\textheight}{1.75in}
% ------------------------------------------------------------------------
% Document Starts here:
% ------------------------------------------------------------------------
\begin{document}
\title{Large-Scale Quantum Chemistry: A Practical Guide for Electronic
Structure Calculations}

\author{Antonio M.\ Ferreira and Henry A. Kurtz}

%\address{Department of Structural Biology\\
%         St. Jude Children's Research Hospital\\
%        262 Danny Thomas Place, Mail Stop 311\\
%         Memphis, Tennessee 38105}

%\address{Department of Chemistry\\
%        The University of Memphis\\
%        Memphis, Tennessee\hspace{1.5ex}38152}\maketitle


% OK >>> Here's a comment to see if everything's now working with CVS.
% Let me know if you can read this, Boss.

This is an attempt to start collecting the ideas for each chapter and
what we may need.  It serves as an organization of ideas mostly.  It
is also a good way for me to relearn to use TeX and the software
again.  I can use the practice.I assume that best way to proceed will
to make each chapter a different file and work on them separately.  In
fact on the publisher's web page of instructions they want each
chapter as aseparate file it looks like.  We could divide them to do
the first drafts.  One start would be: Tony - 4,5; Henry - 2,3.  We
should talka bit more about 1, 6, and 7.  The sections are based on
what was inthe proposal and I expect they will change a lot as it
goes. It seems they like LaTex and take the results as PDF files.
There is a template we can get after they make an agreement with us.

%CHAPTER 1
\chapter{Why Bigger is Better}
\section{The role of modern computer architecture}

\section{The convergence of theory and experiment: the mesocopic world}
Experiments are probing smaller but still a need for larger and larger
theory to actually match.

\section{Quantum effects in biology}
Need a discussion of motivation.  That computational biochemistry
islargely non-electron and empirical.  Can do better than
that,particularly if interest in reactions not just structures.

\section{Qualitative versus Quantitative Descriptions}

%CHAPTER 2
\chapter{A Brief Review of Quantum Chemistry}
\section{Hartree-Fock Theory}

\subsection{Basis Set Discussion}
\subsection{Inclusion of Correlation}
\section{Density Functional Theory}
\section{Semiempirical Quantum Chemistry}
Short discussion of popular schemes
Mozeyme

\section{Approaches to Large Molecules}

\subsection{Fragment-based Methods}
\subsection{Linear-Scaling Methods}
\subsection{The ONIOM method}
\subsection{Local Orbitals versus Canonical Orbitals}
\subsection{Plane Wave Approximations}
\subsection{NBO Analysis}
\section{Something about Molecular Mechanics?}

%CHAPTER 3
\chapter{A Primer for Condensed Matter Physics}
\section{Solvent Effects in Electronic Structure Calculations}
\section{Density of States}
\section{The Kronig-Penny Model}
\section{Semiconductors}
\section{The Tight-Binding Model}
\section{Band Structure}

% CHAPTER 4
\chapter{Electronic Structure for Nucleic Acid Systems}
First part is background on nature of nucleic acid structures and into
to questions and previous calculations.Need List of calculations done
that will be included and discussed.

\section{Start Small:  Early Studies of DNA Bases}
\section{Large-scale Calculations of Electronic Structure}
\subsection{Convergence Issues}
\subsection{Linear Scaling Approximations}
\subsection{Fragment Based Treatments}
\section{Case Study: Ecteinascidin 743}
\subsection{How Big is Big Enough}
\subsection{It's all about the orbitals}

%CHAPTER 5
\chapter{Electronic Structure for Amino Acid Systems}
\section{Why Proteins Differ from DNA: An Electronic Structure Perspective}
\subsection{Nitrogen as an insulator?}
\section{Taking a Big Bite}
\section{Convergence Characteristics}
\section{Comparison of Linear Scaling and FMO Methods}
\section{Case Study: Dihydopteroate Synthase}
\subsection{When do you really need all the protein}
\subsection{Orbital Steering and Quantum Chemistry}
Need List of calculations done that will be included and discussed.

%CHAPTER 6
\chapter{Electronic Structure for Semiconductor Systems}
\section{Periodic Treatments}
\section{Can we get Band Stricture Directly?}
\section{What do we lose in the approximations}
\section{Case Study: ?????}
Options:SiO2 systemsanother system

%CHAPTER 7
\chapter{General Considerations}
\section{Asking the Right Questions}
\section{What Do We Lose With Approximate Methods}
\section{Is Bigger Always Better?}
\section{How Large Can We Go?}

\begin{thebibliography}{99}
\bibitem{GAMESS} M. W. Schmidt, K. K. Baldridge, J. A. Boatz,
S. T. Elbert, M. S. Gordon, J. H. Jensen, S. Koseki, N. Matsunaga,
K. A. Nguyen, S. J. Su, T. L. Windus, M. Dupuis, and J. A. Montgomery
\textit{J.Comput.Chem.}  \textbf{14}, 1347 (1993).

\bibitem{SBKpot} W. J. Stevens, M. Krauss, H. Basch, and P. G. Jasien,
\textit{Can. J.  Chem.} \textbf{70}, 612 (1992); T. R. Cundari and
W. J. Stevens, \textit{J. Chem.  Phys.} \textbf{98}, 5555
(1993).
\end{thebibliography}

\end{document}
